%%%%%%%%%%%%%%%%%%%%%%%%%%%%%%%%%%%%%%%%%
% Plasmati Graduate CV
% LaTeX Template
% Version 1.0 (24/3/13)
%
% This template has been downloaded from:
% http://www.LaTeXTemplates.com
%
% Original author:
% Alessandro Plasmati (alessandro.plasmati@gmail.com)
%
% License:
% CC BY-NC-SA 3.0 (http://creativecommons.org/licenses/by-nc-sa/3.0/)
%
% Important note:
% This template needs to be compiled with XeLaTeX.
% The main document font is called Fontin and can be downloaded for free
% from here: http://www.exljbris.com/fontin.html
%
%%%%%%%%%%%%%%%%%%%%%%%%%%%%%%%%%%%%%%%%%

%----------------------------------------------------------------------------------------
%	PACKAGES AND OTHER DOCUMENT CONFIGURATIONS
%----------------------------------------------------------------------------------------

\documentclass[a4paper,10pt]{article} % Default font size and paper size

\usepackage[utf8]{inputenc}
\usepackage[french]{babel}
\usepackage[T1]{fontenc}

\usepackage{fontspec} % For loading fonts
\defaultfontfeatures{Mapping=tex-text}
\setmainfont[SmallCapsFont = Fontin SmallCaps]{Fontin} % Main document font

\usepackage{xunicode,xltxtra,url,parskip} % Formatting packages

\usepackage[usenames,dvipsnames]{xcolor} % Required for specifying custom colors

%\usepackage[big]{layaureo} % Margin formatting of the A4 page, an alternative to layaureo can be \usepackage{fullpage}
% To reduce the height of the top margin uncomment: \addtolength{\voffset}{-1.3cm}
\usepackage{fullpage}

\usepackage{hyperref} % Required for adding links	and customizing them
\definecolor{linkcolour}{rgb}{0,0.2,0.6} % Link color
\hypersetup{colorlinks,breaklinks,urlcolor=linkcolour,linkcolor=linkcolour} % Set link colors throughout the document

\usepackage{titlesec} % Used to customize the \section command
\titleformat{\section}{\Large\scshape\raggedright}{}{0em}{}[\titlerule] % Text formatting of sections
\titlespacing{\section}{0pt}{3pt}{3pt} % Spacing around sections

\usepackage{enumitem} % Used to reduce the space between items in itemize
\usepackage{tabularx} % Used to customize tabular.
\begin{document}

\pagestyle{empty} % Removes page numbering

\font\fb=''[cmr10]'' % Change the font of the \LaTeX command under the skills section

%----------------------------------------------------------------------------------------
%	NAME AND CONTACT INFORMATION
%----------------------------------------------------------------------------------------


\par{\hfill\centering{\Huge Hadrien \textsc{Renaud}}\hfill
\begin{tabular}{r}
\href{mailto:hadrien.renaud@polytechnique.edu}{hadrien.renaud@polytechnique.edu}\\
+33 6 72 79 70 15
% Appartement 10.20.52\\
%89 boulevard des Maréchaux \\
%91120 Palaiseau
%FRANCE
\end{tabular}
\bigskip\par} % Your name
\vfill

%----------------------------------------------------------------------------------------
%	PERSONNAL PROFILE
%----------------------------------------------------------------------------------------

Un étudiant compétent et adaptable avec 6 mois d'expérience dans le développement de projets informatiques.
Excellentes capacités en mathématiques et informatique.
Cherche un stage dans une entreprise du numérique.

%----------------------------------------------------------------------------------------
%	WORK EXPERIENCE
%----------------------------------------------------------------------------------------

\section{Experience professionnelle}

\begin{tabularx}{\linewidth}{r|X}
  \emph{Maintenant} & Kholleur au \textbf{Lycée Henri IV} à Paris.\\
  \textsc{Sept 2018} & \small{Exercices de préparation aux concours des Grandes Écoles d'Ingénieur. 1h par semaine.}
  \\&\\

\textsc{Oct 2017} & Stage à la \textbf{Direction Générale de l'Armement}\\
\textsc{Avr 2018} & \small{Centre DGA Maîtrise NRBC, Vers-le-Petit (91)
\begin{itemize}[noitemsep, nolistsep, leftmargin=0.5cm]
  \item Développement d'un projet de simulation numérique des risques subis par un porteur d'EPI.
  \item Analyse d'expériences. Expertise sur les méthodes scientifiques utilisées pour tester les EPI.
  \item Conduite d'un projet avec une équipe de 3 personnes.
\end{itemize}}\\

%------------------------------------------------

\textsc{Juin 2016} & Stage à \textbf{Réseau de Transport d'Électricité}, Paris. \\
& \small{Archivage de fichiers, et développement de macros Excel pour le faire à ma place.}
\end{tabularx}

%----------------------------------------------------------------------------------------
%	EDUCATION
%----------------------------------------------------------------------------------------

\section{Formation}

\begin{tabularx}{\linewidth}{r|X}
\emph{Maintenant} & Élève de deuxième année à l'\textbf{École Polytechnique}, Paris, France \\
\textsc{Sept} 2017 & \small{Spécialisation en informatique. Autre matière principale : mathématiques théoriques et appliquées.
Autres matières étudiées : physique quantique, biologie, économie. Projet de groupe : développement d'un générateur aléatoire de scénarios macroéconomiques (en partenariat avec la BNP).
GPA : $3.73/4$.}
\\ & \\
%------------------------------------------------

\textsc{Juil 2017} & Classe préparatoire scientifique au \textbf{Lycée Henri IV}, Paris\\
\textsc{Sept 2015} & \small{Filière MPI (Mathématiques Physique et Informatique)}.\\
  & \small{Travaux de groupe: "Neural networks for caracters recognition".}
\\&\\
%------------------------------------------------

\textsc{Juin 2015} & Baccalauréat scientifique au \textbf{Lycée Henri IV}, Paris\\
\textsc{Sept 2012} & \small{Double spécialité Mathématiques/Informatique. Mention Bien.}
\end{tabularx}

%----------------------------------------------------------------------------------------
%	LANGUAGES
%----------------------------------------------------------------------------------------

\section{Langues}

\begin{tabular}{rl}
  \textsc{Français:} & Natif \\
  \textsc{Anglais:} & C1\\
  \textsc{Allemand:} & B2\\
\end{tabular}

%----------------------------------------------------------------------------------------
%	INTERESTS AND ACTIVITIES
%----------------------------------------------------------------------------------------

\section{Loisirs et autres activités}

\begin{itemize}[noitemsep]
  \item Binet Réseau : Association d'étudiants pour apporter aux étudiants des services informatiques.
  \item Bénévole à l'association La Chorba qui aide les sans-abris à Paris.
  \item Sports : Natation (6h par semaine), footing.
  \item Piano (8ans de concervatoire), lecture, films.
\end{itemize}


%----------------------------------------------------------------------------------------
%	COMPUTER SKILLS
%----------------------------------------------------------------------------------------

\section{Computer Skills}

\begin{tabular}{rl}
  Expert: & Python\\
  Bonne maîtrise: & Java, Linux, MySQL, C/C++\\
  Connaissance intermédiaires: & Web (HTML, CSS, JS, Cordova), \LaTeX, VBA, OCaml
\end{tabular}



\end{document}
